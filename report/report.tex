\documentclass{article}

\usepackage[english]{babel}

\usepackage[letterpaper,top=2cm,bottom=2cm,left=3cm,right=3cm,marginparwidth=1.75cm]{geometry}

\usepackage{amsmath}
\usepackage{graphicx}
\usepackage[colorlinks=true, allcolors=blue]{hyperref}
\usepackage{natbib}
\bibliographystyle{alpha}
\usepackage{caption}
\usepackage{float}

\title{Redes de Computadores \\ \large Trabalho Prático 2}
\author{Luís Felipe Ramos Ferreira}
\date{\href{mailto:lframos\_ferreira@outlook.com}{\texttt{lframos\_ferreira@outlook.com}}
}

\begin{document}

\maketitle

\section{Introdução}

O Trabalho Prático 2 da disciplina de Sistemas operacionais teve como proposta
o estudo e modificação dos algoritmos de escalonamento de processos presente
no \textit{kernel} do sistema operacional
\href{https://github.com/mit-pdos/xv6-public}{XV6}.

O repositório onde está armazenado o código utilizado durante o desenvolvimento
desse projeto
pode ser encontrado \href{https://github.com/lframosferreira/tp2-so}{neste
      endereço}.

\section{Respostas}

\begin{enumerate}
      \item Qual a política de escalonamento é utilizada atualmente no XV6?
      \item Quais processos essa política seleciona para rodar?
      \item O que acontece quando um processo retorna de uma tarefa de I/O?
      \item O que acontece quando um processo é cirado e quando ou quão
            frequente o escalonamento acontece?
\end{enumerate}

\subsection{Algoritmos implementados}

Por s

\section{Análise de resultados}

A p

\section{Conclusão}

Em s

\section{Referências}

\begin{itemize}
      \item Livros:
            \begin{itemize}
                  \item Tanenbaum, A. S. \& Bos, H. (2014), Modern Operating
                        Systems, Pearson, Boston, MA.
                  \item Abraham Silberschatz, Peter Baer Galvin, Greg Gagne:
                        Operating System Concepts, 10th Edition. Wiley 2018, ISBN
                        978-1-118-06333-0
                  \item Arpaci-Dusseau, Remzi H., Arpaci-Dusseau, Andrea C.. (2014).
                        Operating systems: three easy pieces.: Arpaci-Dusseau Books.
            \end{itemize}

      \item Web:
            \begin{itemize}
                  \item
            \end{itemize}

      \item Youtube:
            \begin{itemize}
                  \item \href{https://www.youtube.com/@JacobSorber}{Jacob
                              Sorber}
                  \item \href{https://www.youtube.com/@CodeVault}{Code Vault}
            \end{itemize}

\end{itemize}

\end{document}