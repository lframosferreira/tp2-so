\documentclass{article}

\usepackage[english]{babel}

\usepackage[letterpaper,top=2cm,bottom=2cm,left=3cm,right=3cm,marginparwidth=1.75cm]{geometry}

\usepackage{amsmath}
\usepackage{graphicx}
\usepackage[colorlinks=true, allcolors=blue]{hyperref}
\usepackage{natbib}
\bibliographystyle{alpha}
\usepackage{caption}
\usepackage{float}

\title{Sistemas Operacionais \\ \large Trabalho Prático 2}
\author{Luís Felipe Ramos Ferreira}
\date{\href{mailto:lframos\_ferreira@outlook.com}{\texttt{lframos\_ferreira@outlook.com}}
}

\begin{document}

\maketitle

\section{Introdução}

O Trabalho Prático 2 da disciplina de Sistemas operacionais teve como proposta
o estudo e modificação dos algoritmos de escalonamento de processos presente
no \textit{kernel} do sistema operacional
\href{https://github.com/mit-pdos/xv6-public}{XV6}.

O repositório onde está armazenado o código utilizado durante o desenvolvimento
desse projeto
pode ser encontrado \href{https://github.com/lframosferreira/tp2-so}{neste
      endereço}.

\section{Respostas}

\begin{enumerate}
      \item Qual a política de escalonamento é utilizada atualmente no XV6?

            A política de escalonamento utilizada no XV6 é uma política de
            \href{https://en.wikipedia.org/wiki/Round-robin_scheduling}{\textit{Round
                        Robin}}, ou seja, o escalonador irá checar
            continuamente a lista de processos disponíveis para serem
            executados e irá
            fornecer um tempo de processamento a cada um deles. O algoritmo é
            simples de se implementar, simples de compreender e não causa
            inanição aos
            processos, embora possua pontos negativos como os gargalos causados
            pela
            constante troca de contexto, a depende do tempo que cada processo
            terá para
            execução na CPU. É um algoritmo preemptivo, uma vez que força a
            saída de um processo da CPU caso o limite de tempo tenha sidoa
            tingido.\@

      \item Quais processos essa política seleciona para rodar?

            A política citada seleciona os processos que estão disponíveis para
            serem executados conforme eles são checados na lista de processos
            disponíveis.
            Não há um tipo de prioridade estabelecida em cima sobre os
            processos, ou seja, os processos terão uma certa quantidade de
            tempo a cada
            momento que o escalonador encontrá-los na lista de processos
            disponíveis. É
            importante frisar que o escalonador irá checar \texttt{apenas} os
            processos marcados como disponíveis para serem executados, ou seja,
            processos dormindo ou esperando algum I/0 não receberão tempo de
            processamento
            da CPU a menos que estejam prontos para serem executados e marcados
            como tal na
            lista de processos do sistema.

      \item O que acontece quando um processo retorna de uma tarefa de I/O?

            O processo é marcado como \textit{RUNNABLE}, isto é, está pronto
            para executar e entra para a lista de processos que podem ser
            executados.
            Assim, ele ventualmente será escolhido pelo escalonador para
            começar a rodar. ISSO MSM?

      \item O que acontece quando um processo é criado e quando ou quão
            frequente o escalonamento acontece?

            Quando um processo é criado, uma referência para ele é
            criada no espaço de memória do sistema operacional, e esse novo
            processo deve possuir alocado para ele um espaço de memória de
            usuário onde irá
            estar armazenado seu identificador, código, dados, pilha de
            execução e \textit{heap}. Um processo pode ser criado no XV6 por
            meio da chamada de sistema \textit{fork()}, que irá criar uma cópia
            do processo que o criou. Para executar um novo programa, a chamada
            de sistema
            \textit{exec()}
            deve ser utilizada.

            Não sei de quanto em quanto tempo, mas provavelmente a cada 1 tick
            do clock como dito no proprio enunciado logo abaixo.
\end{enumerate}

\subsection{Algoritmos implementados}

Por s

\section{Análise de resultados}

A p

\section{Conclusão}

Em s

\section{Referências}

\begin{itemize}
      \item Livros:
            \begin{itemize}
                  \item Tanenbaum, A. S. \& Bos, H. (2014), Modern Operating
                        Systems, Pearson, Boston, MA.
                  \item Abraham Silberschatz, Peter Baer Galvin, Greg Gagne:
                        Operating System Concepts, 10th Edition. Wiley 2018,
                        ISBN
                        978-1-118-06333-0
                  \item Arpaci-Dusseau, Remzi H., Arpaci-Dusseau, Andrea C..
                        (2014).
                        Operating systems: three easy pieces.: Arpaci-Dusseau
                        Books.
            \end{itemize}

      \item Web:
            \begin{itemize}
                  \item

                        \href{https://pdos.csail.mit.edu/6.828/2023/xv6/book-riscv-rev3.pdf}{\textit{xv6:
                                    a
                                    simple, Unix-like teaching operating
                                    system}}
            \end{itemize}

      \item Youtube:
            \begin{itemize}
                  \item \href{https://www.youtube.com/@JacobSorber}{Jacob
                              Sorber}
                  \item \href{https://www.youtube.com/@CodeVault}{Code Vault}
                  \item

                        \href{https://www.youtube.com/watch?v=fWUJKH0RNFE&list=PLbtzT1TYeoMhTPzyTZboW_j7TPAnjv9XB}{hhp3
                              xv6 \textit{kernel playlist}}
            \end{itemize}

\end{itemize}

\end{document}